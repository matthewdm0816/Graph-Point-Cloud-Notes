\documentclass{article}
\usepackage{ctex, xeCJK}
\usepackage{metalogo, makecell, svg, amssymb, amsfonts, amsmath, physics, fancyhdr, geometry, graphicx, pdfpages, ragged2e, bm}
\newcommand{\R}{\mathbb{R}}
\newcommand{\rarr}{\rightarrow}
\newcommand{\lop}{Laplace算子}
\newcommand{\tRarr}{$\Rightarrow$}
\newcommand{\trarr}{$\Rightarrow$}
\newcommand{\filter}{\Gamma_{l,l'}}
\newcommand{\fn}[1]{\footnote{#1}}
\newcommand{\bs}[1]{\boldsymbol{#1}}
\newcommand{\iprod}[2]{\langle #1, #2 \rangle}
\newcommand{\define}{\textbf{Definition} }
\newcommand{\trm}{\textbf{Theorem} }
\newcommand{\alg}{\textbf{Algorithm} }
\newcommand{\cov}{\text{Cov}}
\newcommand{\bb}{\mathbb}
\newtheorem{theorem}{Theorem}[section]
\newtheorem{lemma}[theorem]{Lemma}
\newtheorem{proposition}[theorem]{Proposition}
\newtheorem{corollary}[theorem]{Corollary}
\newtheorem{definition}[theorem]{Definition}

\newenvironment{proof}[1][Proof]{\begin{trivlist}
\item[\hskip \labelsep {\bfseries #1}]}{\end{trivlist}}
% \newenvironment{definition}[1][Definition]{\begin{trivlist}
% \item[\hskip \labelsep {\bfseries #1}]}{\end{trivlist}}
\newenvironment{example}[1][Example]{\begin{trivlist}
\item[\hskip \labelsep {\bfseries #1}]}{\end{trivlist}}
\newenvironment{idea}[1][Idea]{\begin{trivlist}
\item[\hskip \labelsep {\bfseries #1}]}{\end{trivlist}}
\newenvironment{remark}[1][Remark]{\begin{trivlist}
\item[\hskip \labelsep {\bfseries #1}]}{\end{trivlist}}
\renewcommand{\cal}{\mathcal}
\usepackage[hidelinks, bookmarks]{hyperref} % add index hyper-links
\newcommand{\coro}{\textbf{Corollary} }
\newcommand{\tgt}{\textbf{Target} }
\newcommand{\bt}[1]{\textbf{#1}}
\newcommand{\lp}{Lagrange Polynomial}
\newcommand{\np}{Newton Polynomial}
\newcommand{\where}{\text{where }}
\newcommand{\centerimage}[2]{
    \centerline{\includegraphics[width=#1\paperwidth]{#2}
    }
}
\let\titleoriginal\title           % save original \title macro
\newcommand{\thetitle}{}
\renewcommand{\title}[1]{          % substitute for a new \title
    \titleoriginal{#1}%               % define the real title
    \renewcommand{\thetitle}{#1}        % define \thetitle
}
\title{\textbf{Notes on 音乐与数学}}
\author{Matthew Mo}
\date{\today}
% set plain style
\pagestyle{fancy}
\lhead{} 
\chead{} 
\rhead{} 
\lfoot{\it Notes on Geometric Learning} 
\cfoot{}
\rfoot{\thepage} 

% Length to control the \fancyheadoffset and the calculation of \headline
% simultaneously
\newlength\FHoffset
\setlength\FHoffset{1cm}

\addtolength\headwidth{2\FHoffset}

\fancyheadoffset{\FHoffset}

% these lengths will control the headrule trimming to the left and right 
\newlength\FHleft
\newlength\FHright

% here the trimmings are controlled by the user
\setlength\FHleft{1cm}
\setlength\FHright{0cm}

% The new definition of headrule that will take into acount the trimming(s)
\newbox\FHline
\setbox\FHline=\hbox{\hsize=\paperwidth%
  \hspace*{\FHleft}%
  \rule{\dimexpr\headwidth-\FHleft-\FHright\relax}{\headrulewidth}\hspace*{\FHright}%
}
\renewcommand\headrule{\vskip-.7\baselineskip\copy\FHline}

\renewcommand{\headrulewidth}{0.7pt} % hline width
\renewcommand{\footrulewidth}{0.7pt} 

% set margin
\geometry{a4paper,scale=0.75}
\newcommand{\note}{\textbf{Note} }
\begin{document}
\maketitle
\section*{ }
\label{contents}
\tableofcontents

\section{Basics}

    和谐的音程: \begin{itemize}
        \item 1:1 ~ 同度
        \item 2:1 ~ 八度
        \item 3:2 ~ 纯五度
        \item 4:3 ~ 纯四度
        \item 5:4 ~ 大三度(理想)
    \end{itemize}

\subsection{节奏:音乐的时间形式}
    音乐的要素: 节奏, 旋律, 和声
    
    \bt{Def.} 节奏是乐音时值的有组织的顺序\dots

    \bt{Note} 区分节拍和节奏
    
    \bt{Def.} 节拍... 2/4 etc.

    \bt{Def.} 固定节奏型(rythmic orisnato). ``拍''(pulse):十六分音符.

    \bt{Prop.} 极大均衡原则: 在所有的拍上,起拍应该尽量均匀.

    \bt{Def.} ``圆周上的几何节奏模型''

    \bt{Prop.} 节奏奇性: 圆周上起拍对位不是起拍.

    \bt{Def.} 距离序列: 起拍之间的间隔序列.

    听觉系统对绝对时间差相对不敏感, 对相对变化敏感 \trarr

    \bt{Def.} 轮廓(Contour):节奏间隔变化的序列; 轮廓同构: 若节奏型可以通过循环轮换等价\dots 

    \bt{Def.} 影子节奏. 古巴颂节奏轮廓同构于影子节奏轮廓!
\subsection{旋律与对称}
    \bt{Def.} 旋律(melody): 不同音高组成的序列. *旋律应遵循对称原则
 
    重复与变化!

    \bt{旋律的变化:移调(transposition)} 分为\bt{严格移调}和\bt{调性移调}.

    \bt{旋律的变化:逆行(retrograde), 倒影(...)}
    
    这些都是旋律的某种对称. 传统调性音乐对这些对称的应用进行限制.

    对称是美的重要来源. 群论用于研究对称.

\section{成绩}

成绩评定:平时 20' 期中研究 30' 期末闭卷考试 50'

教材:音乐与数学.

``基本乐理通用教材,李重光'', ``数学与音乐,周明儒''

个人网页 | 北京大学电子教参平台

``TheEnjoymentofMusic - A Glossory''

\section{音乐基本知识}
\subsection{音乐的诸要素}
    音乐:\dots .声音是音乐的载体, 声波是纵波. 
    音乐的要素: 
    \begin{enumerate}
        \item 频率\trarr 音高(pitch);
        \item 空气压力 \trarr 力度;
        \item 时间长度 \trarr 时值(durations);
        \item 波形 \trarr 音色(timbre).
    \end{enumerate}   

    音乐会音高 440Hz - Center C

    力度: 20微帕=0dB@1kHz. SPL:$L_p=20\log (p/p_0)$. HRTF/等响曲线

    音色: 泛音(overtone), 波形(waveform), 频谱(spectrum)决定了音色.

    乐音体系: 乐音(musical tone)与噪音(noise). 有组织的噪音(organized noise). 打击乐器中也有有固定音高(木琴,定音鼓)和无固定音高的乐器(嚓, 锣, 小军鼓). 每一个音有一个音名(pitch name): CDEFGAB, 是基本音级. 在每个八度(octave)里循环使用, 分为了若干音组. 变音符号 \trarr 变化音级(升降号, 重升降号, 还原号). 异名同音:两个音名有相同音高 \trarr 等音的(enharmonic). 唱名:do re mi fa so la si, mi-fa, si-do之间是半音. 首调唱名法(movable do), $1=X$.
\subsection{乐谱}
    
    记谱法(notation), 五线谱. 非线性坐标系! 音符: 符头, 符干, 符尾. 音符代表的时值是相对长度 ~ 取决于拍子. 谱号(clef), 常用的有高音谱号(二线为$G_4$), 低音谱号(冒号中心为四线$F_3$), 这二者的上/下加一线为中央C; 中音谱号(次中音谱号...). 谱号+五线谱 \trarr 谱表(staff). 
    变音记号: 谱号后面的调号(key signature), 在未改变调号之前都生效.

    音程: 两个音级之间的距离称为音程(interval), 上方音(冠音)-下方音(根音); 度数(五线谱上的线和间的个数)+半音数决定了音程! 小二度(半音数为1),大二度(半音数为2); 小三度(和谐? 半音数为3), 大三度(半音数为4);纯四度(半音数为5, 和谐!), 增四度(半音数为6, 三全音); 减五度(半音数为6), 纯五度(半音数为7, 和谐?); 小七度(11), 纯八度(12), 纯一度(0). 自然音程(diatonic interval): 两边都是白键, 自然音程增减半音变为变化音程.大/纯增一个半音为增音程, 小/纯减一个半音为减音程. 
    
    (不)协和(consonant)音程: 纯五度最和谐开始; 和谐: 纯四/五/八(完全协和), 大小三/六度(不完全协和), 其他不协和.
    
    练习: 练耳程序-协和程度 | 预习教材第二章

    泛音列重合理论.

\section{音乐的数学理论}
\begin{flushleft}
    乐器的分类: 气鸣乐器, 弦鸣乐器, 电鸣乐器, 体鸣乐器, 膜鸣乐器.

    振动满足Laplace初值问题
    \begin{equation}
        \begin{aligned}
            &\frac{\partial^2 u}{\partial t^2} = c^2 \sum_{i} \frac{\partial^2 u}{\partial x_i^2}\\
            &u|_{\partial \Omega} = 0
        \end{aligned}
    \end{equation}
    一维振动方程的解
    \begin{equation}
        \begin{aligned}
        u(x, t) &=\sum_{n=1}^{\infty} u_{n}(x, t) \\
        &=\sum_{n=1}^{\infty} \sin \left(\frac{n \pi}{L} x\right)\left(a_{n} \cos \frac{n \pi c}{L} t+b_{n} \sin \frac{n \pi c}{L} t\right)
        \end{aligned}
    \end{equation}

    \begin{theorem}
        (Mersenne) $$f_1=\frac{1}{2L}\sqrt{\frac{T}{\rho}}$$
    \end{theorem}

    称$f_1$为基音, $f_n$为$n-1$泛音, 都是基频的倍频, 总共称为泛音列.

    拨弦振动(L/2处)的Fourier级数阶 \trarr 只有奇次谐波!
    \begin{equation}
        \begin{aligned}
        u(x, t) &=\sum_{n=1}^{\infty} a_{n} \sin \left(\frac{n \pi}{L} x\right) \cos \left(\frac{n \pi c}{L} t\right)(1) \\
        &=\sum_{k=0}^{\infty}(-1)^{k} \frac{8}{(2 k+1)^{2} \pi^{2}} \sin \left(\frac{(2 k+1) \pi}{L} x\right) \cos \left(\frac{(2 k+1) \pi c}{L} t\right)
        \end{aligned}
    \end{equation}

    管乐器的空气柱边界条件不确定, 往往调音(端口校正, end correction)需要靠听. 不计端口校正, 开口处为波腹, 闭口处必为波节. 故开管乐器$\lambda_n=\frac{2L}{n}, f_1=\frac{v}{2L}, f_n=nf_1$, 闭管乐器$\lambda_n=\frac{4L}{n}, n\text{为奇数}, f_1=\frac{v}{2L}, f_n=nf_1$, 只有奇数次谐波


\end{flushleft}
\end{document}