\documentclass{article}
\title{\textbf{Notes on 信号与系统}}
\author{Matthew Mo}
\date{\today}

\usepackage{ctex, xeCJK, zxjatype}
\usepackage{metalogo, makecell, svg, amssymb, amsfonts, amsmath, physics, fancyhdr, geometry, graphicx, pdfpages, ragged2e, bm}
\usepackage{verbatim}
\newcommand{\R}{\mathbb{R}}
\newcommand{\rarr}{\rightarrow}
\newcommand{\lop}{Laplace算子}
\newcommand{\tRarr}{$\Rightarrow$}
\newcommand{\trarr}{$\Rightarrow$}
\newcommand{\filter}{\Gamma_{l,l'}}
\newcommand{\fn}[1]{\footnote{#1}}
\newcommand{\bs}[1]{\boldsymbol{#1}}
\newcommand{\iprod}[2]{\langle #1, #2 \rangle}
\newcommand{\define}{\textbf{Definition} }
\newcommand{\trm}{\textbf{Theorem} }
\newcommand{\alg}{\textbf{Algorithm} }
\newcommand{\cov}{\text{Cov}}
\newcommand{\bb}{\mathbb}
\newtheorem{theorem}{Theorem}[section]
\newtheorem{lemma}[theorem]{Lemma}
\newtheorem{proposition}[theorem]{Proposition}
\newtheorem{corollary}[theorem]{Corollary}
\newtheorem{definition}[theorem]{Definition}

\newenvironment{proof}[1][Proof]{\begin{trivlist}
\item[\hskip \labelsep {\bfseries #1}]}{\end{trivlist}}
% \newenvironment{definition}[1][Definition]{\begin{trivlist}
% \item[\hskip \labelsep {\bfseries #1}]}{\end{trivlist}}
\newenvironment{example}[1][Example]{\begin{trivlist}
\item[\hskip \labelsep {\bfseries #1}]}{\end{trivlist}}
\newenvironment{idea}[1][Idea]{\begin{trivlist}
\item[\hskip \labelsep {\bfseries #1}]}{\end{trivlist}}
\newenvironment{remark}[1][Remark]{\begin{trivlist}
\item[\hskip \labelsep {\bfseries #1}]}{\end{trivlist}}
\renewcommand{\cal}{\mathcal}
\usepackage[hidelinks, bookmarks]{hyperref} % add index hyper-links
\newcommand{\coro}{\textbf{Corollary} }
\newcommand{\tgt}{\textbf{Target} }
\newcommand{\bt}[1]{\textbf{#1}}
\newcommand{\lp}{Lagrange Polynomial}
\newcommand{\np}{Newton Polynomial}
\newcommand{\where}{\text{where }}
\newcommand{\centerimage}[2]{
    \centerline{\includegraphics[width=#1\paperwidth]{#2}
    }
}
\let\titleoriginal\title           % save original \title macro
\newcommand{\thetitle}{}
\renewcommand{\title}[1]{          % substitute for a new \title
    \titleoriginal{#1}%               % define the real title
    \renewcommand{\thetitle}{#1}        % define \thetitle
}

\newCJKfontfamily\gothic{IPAexGothic}
\newCJKfontfamily\mincho{IPAexMincho}

% set plain style
\pagestyle{fancy}
\lhead{} 
\chead{} 
\rhead{} 
% \lfoot{\it Notes on Geometric Learning} 
\lfoot{\hyperref[contents]{\thetitle}} 
% \lfoot{\thetitle} 
\cfoot{}
\rfoot{\thepage} 


% Length to control the \fancyheadoffset and the calculation of \headline
% simultaneously
\newlength\FHoffset
\setlength\FHoffset{1cm}

\addtolength\headwidth{2\FHoffset}

\fancyheadoffset{\FHoffset}

% these lengths will control the headrule trimming to the left and right 
\newlength\FHleft
\newlength\FHright

% here the trimmings are controlled by the user
\setlength\FHleft{1cm}
\setlength\FHright{0cm}

% The new definition of headrule that will take into acount the trimming(s)
\newbox\FHline
\setbox\FHline=\hbox{\hsize=\paperwidth%
  \hspace*{\FHleft}%
  \rule{\dimexpr\headwidth-\FHleft-\FHright\relax}{\headrulewidth}\hspace*{\FHright}%
}
\renewcommand\headrule{\vskip-.7\baselineskip\copy\FHline}

\renewcommand{\headrulewidth}{0.7pt} % hline width
\renewcommand{\footrulewidth}{0.7pt} 

% set margin
\geometry{a4paper,scale=0.75}
\newcommand{\note}{\textbf{Note} }
\begin{document}
\maketitle
\tableofcontents

\section{Basics}

    \bt{Contents: Signal Proc.}
    \begin{enumerate}
        \item Intro to Signal Processing
        \item 线性时不变系统时域分析/Temporal Ana. of Linear Time-Invariant System
        \item Fourier Analysis
        \item Laplace Transform
        \item Z Transform
        \item DFT and FFT
    \end{enumerate}
    
    \bt{Contents: System}
    \begin{enumerate}
        \item 线性时不变系统分析
        \item 数字滤波器设计
        \item 数字滤波器结构
        \item 有限字长效应的统计分析
    \end{enumerate}

    \bt{Grades}
    \begin{itemize}
        \item 出席 5' 课堂讨论 5' 
        \item 作业 30'
        \begin{itemize}
            \item 课后网上发布, 一周后随堂提交.
        \end{itemize}
        \item 考试 60'
    \end{itemize}

\section{Signals}

\subsection{Common Continous Signals}

    \bt{Exponential} \begin{align}
        &f(t)=K \exp(at)\\
        &\where \tau = \frac{1}{|a|}\text{是时间常数}
    \end{align}
 
    \bt{Sine} \begin{align}
        &f(t)=K \sin (\omega t+\theta)\\
        &\where T = \frac{2\pi}{\omega}s
    \end{align}

    \bt{Complex Exponential} \begin{align}
        &f(t)=K \exp(st)
    \end{align}

    \bt{Sa/抽样函数} \begin{align}
        f(t) &= \frac{\sin t}{t}\\
        \int_\R f(t)dt&=\pi, \int_{\R_+} f(t)dt=\pi/2
    \end{align}

    \bt{Gaussian} \begin{align}
        &f(t) = E \exp(-\frac{t^2}{\tau^2})
    \end{align}

    \bt{Rectified Linear/ReLU} \begin{align}
        &R(t) = \max(0, t)
        &R(t) = \min(\tau, \max(0, t)) \text{截顶的ReLU}
    \end{align}

    \bt{Unit Leap/Heavside} \begin{align}
        &u(t)=\begin{cases}
            1, t > 0\\
            1/2, t=0 \\
            0, t<0
        \end{cases}
    \end{align}
    
    \bt{Square Impulse} \begin{align}
        &G_\tau(t)=\begin{cases}
            1, t\in (-\tau/2, \tau/2)\\
            1/2, |t|=\tau/2\\
            0, \text{otherwise}
        \end{cases}\\
        &G_\tau(t)=u(t+\tau/2)-u(t-\tau/2)
    \end{align}

    \bt{Sign} \begin{align}
        &sgn(t)=\begin{cases}
            1, t>0\\
            -1, t<0
        \end{cases}\\
        &sgn(t) = 2u(t)-1
    \end{align}

    \bt{Dirac Delta} 一个广义函数/测度. \begin{align}
        &\delta(t) = \begin{cases}
            \infty, t=0\\
            0, \text{otherwise}
        \end{cases}
    \end{align}
    有性质\begin{align}
        &\int_\R \delta(t)dt = 1\\
        &\delta(at) = \frac{1}{|a|} \delta(t)\\
        &\int_\R \delta_{t_0}(t) f(t) dt = f(t_0)
    \end{align}
    他的导数是一对正负冲激\begin{align}
        \int_R \delta'_{t_0}(t) f(t)dt = -f'(t_0)
    \end{align}

\subsection{Discrete Time Signal}

    \bt{Real/Complex Exponential} \begin{align}
        &x[n] = A z^n
    \end{align}

    \bt{Unit Sampling/Leap} \begin{align}
        \delta[n] &= \begin{cases}
            1, n=0\\
            0, n\not=0
        \end{cases}\\
        u[n]&=\begin{cases}
            1, n\ge0\\
            0, n<0
        \end{cases}
    \end{align}

    \bt{Rectangular Seq.} \begin{align}
        &R_N[n] = \begin{cases}
            1, 0\le n < N \\
            0, n\ge N
        \end{cases}
    \end{align}

\subsection{Decompostion of Signals}

    \bt{直流/交流分量} 记信号的平均值为直流分量
    \begin{align}
        &f_{DC} = \lim_{T\to \infty} \frac{1}{T} \int_{t\in [-T/2, T/2]} f(t)dt\\
        &f_{AC} = f - f_{DC}\\
        &P = f_{DC}^2 + \lim_{T\to \infty} \frac{1}{T} \int_{t\in [-T/2, T/2]} f_{AC}(t)^2 dt
    \end{align}

    \bt{奇偶分量} \begin{align}
        &f(t) = g(t) + h(t),\\
        \where &g(t)=\frac{1}{2} (f(t)+f(-t))\\
        &h(t) = \frac{1}{2}(f(t)-f(-t))
    \end{align}

    \bt{R/C分量} \begin{align}
        f(t) = \Re(f(t)) + \Im(f(t))
    \end{align}

    \bt{Impulse/Leap Decomp.} \begin{align}
        f(t)&=\int^t_0 f(\tau) \delta(t-\tau)d\tau\\
        &=f_{[0..t]} * \delta\\
        f(t) &= f(0)u(t) + \int^t_0 f'(\tau) u(t-\tau)d\tau
    \end{align}

    \bt{Orthogonal Decomp.}

\subsection{Classifications of Signals}

    \begin{enumerate}
        \item 连续/离散
        \item 奇偶
        \item 确定性/随机性
        \item 周期/非周期
        \item 因果/反因果/非因果:仅在$\R_+/\R_-/\R$有信号
        \item 实/复信号
        \item 能量/功率有限信号
    \end{enumerate}

\end{document}